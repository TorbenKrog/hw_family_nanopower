	%\documentclass[]{article}
\documentclass[9pt,a4paper]{article}

\usepackage{geometry}
\geometry{verbose,a4paper,tmargin=3cm,bmargin=3cm,lmargin=2cm,rmargin=2cm,headheight=30.0pt}

\usepackage[table]{xcolor}
\definecolor{SkyBlue}{HTML}{3465A4}
\definecolor{DarkSkyBlue}{HTML}{204A87}
\definecolor{Plum}{HTML}{75507B}
\definecolor{ScarletRed}{HTML}{CC0000}
\definecolor{Aluminium1}{HTML}{EEEEEC}
\definecolor{Aluminium6}{HTML}{2e3436}
\definecolor{Black}{HTML}{000000}

\usepackage{sectsty}
\allsectionsfont{\sffamily}

\setlength{\itemsep}{-1pt}
\usepackage{varioref}
\usepackage{alltt}
\usepackage[T1]{fontenc}
\usepackage{lmodern}
\usepackage{amssymb,amsmath}
\usepackage{ifxetex,ifluatex}
\usepackage{fixltx2e} % provides \textsubscript
% use microtype if available
\IfFileExists{microtype.sty}{\usepackage{microtype}}{}
\ifnum 0\ifxetex 1\fi\ifluatex 1\fi=0 % if pdftex
  \usepackage[utf8]{inputenc}
\else % if luatex or xelatex
  \usepackage{fontspec}
  \ifxetex
    \usepackage{xltxtra,xunicode}
  \fi
  \defaultfontfeatures{Mapping=tex-text,Scale=MatchLowercase}
  \newcommand{\euro}{€}
\fi
\usepackage{multirow}

\newcommand{\superscript}[1]{\ensuremath{^{\textrm{#1}}}}

\ifxetex
  \usepackage[setpagesize=false, % page size defined by xetex
              unicode=false, % unicode breaks when used with xetex
              xetex]{hyperref}
\else
  \usepackage[unicode=true]{hyperref}
\fi
\hypersetup{breaklinks=true,
            bookmarks=true,
            pdfauthor={GomSpace ApS},
            pdftitle={NanoPower P31U-7.0 Change Note},
            colorlinks=true,
            urlcolor=DarkSkyBlue,
            linkcolor=DarkSkyBlue,
            pdfborder={0 0 0}}
\setlength{\parindent}{0pt}
\setlength{\parskip}{6pt plus 2pt minus 1pt}
\setlength{\emergencystretch}{3em}  % prevent overfull lines

\usepackage{lastpage} % for the number of the last page in the document
\usepackage{fancyheadings}
\pagestyle{fancy}
\fancyhf{}
\lhead{\parbox{5cm}{\includegraphics{img/gomspace_logo.png}}}
\rhead{\parbox{5cm}{
\begin{tabular}{lr}	
    Date   & \today \\
	Ref:   & GS-CN-P31U70 \\ 
	Title: & NanoPower P31U-7.0 Change Note
\end{tabular}}
}
\lfoot{\copyright\hspace{1mm}2012 GomSpace ApS}
\rfoot{Page \thepage\ of \pageref{LastPage}}
\renewcommand{\footrulewidth}{0.5pt}



\title{NanoPower P31U-7.0 Change Note}
\author{GomSpace ApS}
\date{November 2012}

\begin{document}
%\maketitle

\newcommand{\HRule}{\rule{\linewidth}{0.5mm}}
\begin{titlepage}
\begin{center}
% Upper part of the page
\includegraphics[width=0.3\textwidth]{./img/board.png}\\[1cm]    
\textsc{\LARGE GomSpace ApS}\\[1.5cm]
\textsc{\Large Datasheet}\\[0.5cm]
% Title
\HRule \\[0.4cm]
{ \huge \bfseries NanoPower P31U-7.0 Change Note}\\[0.2cm]
\input{subtitle.tex}
\\
[0.2cm]
\HRule \\[1.5cm]
% 
\begin{minipage}{0.4\textwidth}
\begin{flushleft} \large
\emph{Doc-ref:} \\
GS-CN-P31U70
\end{flushleft}
\end{minipage}
\begin{minipage}{0.4\textwidth}
\begin{flushright} \large
\emph{Git-ref:} \\
NA 
\end{flushright}
\end{minipage}
\vfill
% Bottom of the page
{\large \today}
\end{center}
\end{titlepage}



{
\hypersetup{linkcolor=black}
\newpage
\setcounter{tocdepth}{2}
\tableofcontents
}
\newpage
\section{Design Upgrade Note NP31U-7.0}

The NanoPower P31U hardware design has undergone an iteration taking it
from revision 6.1.1 to revision 7.0. This upgrade is motivated by needs
and wishes regarding both functionality, performance, handling,
integration, producibility and aesthetics. The motivational factors are
listed in the categories below. Besides all these factors, the design is
migrated from Eagle PCB to Altium DXP making future upgrades far less
painful and much more reliable.

\subsection{Functionality}

It has frequently been inquired as to whether the P31U can be configured
with an arbitrary assignment of the two regulated output busses to any
of the six latch-up protected outputs. This is now accommodated through
a connection matrix. This matrix is made up of current sense resistors
giving the added benefit of current measurements on each output channel
which is a desire from the R\&D dept. at GomSpace.

Having the ability to power cycle the entire P31U including internal
supplies and all external channels is a wish from e.g.~FunCube. This is
now accommodated by tripping the kill-switch circuit from the uC.

The temperature sensor previously located at boost converter 2 is not
moved to the measure the temprature of the buck converter's ground
return plane. Experience has shown that there is only little temperature
gradient between the three boost converters making three measurements
somewhat redundant.

\subsection{Performance}

The Maximum Power Point Tracking algorithm has been lacking accuracy and
performance at high tumbling rates due to triple PV input current
estimation using just the one ouput current measurement of the battery
charge converter. This is solved by adding current measurements to all
PV inputs enabling accurate and fast calculation of input power.

Boost converters in the battery charger used high inductances resulting
in high DC loss. This is optimised by lowering inductance and hence
resistance.

The buck converters have also got smaller inductances in order to
optimise effeciency by minimising DC loss.

A more robust latch-up detection scheme is implemented using a dedicated
interrupt channel on the uC for each output. This eliminates a large and
bulky and often unreliable transistor-based detector circuit.

\subsection{Handling}

Some tracks carrying raw battery voltage were very close to mounting
holes in one corner causing a number of product failures due to short
circuits when users were less than extremely careful with washers during
integration. Battery voltage tracks are now kept at a safe distance from
grounded areas.

The arm jumper at the positive battery terminal was located in a hard to
reach spot on the PCB making it dangerous to operate the jumper when the
P31U is integrated in a stack. The jumper is now moved to the edge of
the board in an area containing no obstacles. This is obtained by
rotating the battery pack 180 degrees such that + and - now face away
from the stach connector.

\subsection{Integration}

All connections to and from the board are now through connectors instead
of the through-hole solder pads used by PV inputs, charge input and
kill-switch operation. This makes it much more convenient to integrated
the P31U in a satellite. For the critical PV inputs redundant wiring is
used - four wires per solar panel instead of two.

Two kill-switch connectors are placed at diagonal corners to enable
short kill-switch wires.

Connectors for ISP and serial ports are placed at GomSpace standard
location at one edge of the board to enable programming and debugging of
a P31U even once integrated into a stack.

\subsection{Producibility}

All through-hole components are replaced by SMD to enable 99\% automated
assembly.

Outer layer ground planes and supply planes are reshaped to enable
reflow soldering with automatic component alignment. This is done by
using symmetric fills and by placing vias outside large solder areas
such as thermal pads.

The inter-battery cell tab is ommitted by adding this connection on the
PCB. This means that alle LiIon cells used in GomSpace products can now
have the same tab configuration enabling mass batch production of tabbed
batteries. This is possible because of the 180 degree rotation of the
cells. It adds the benefit of safer soldering of the tabs to the board
as three of the tabs are using slotted through holes similar to the BP4
system which minimises the risk of short circuits (a.k.a. arc welding)
during soldering.

The use of connectors instead of through-hole pads for all external
connections makes harness construction much easier as we can make cable
assemblies of pre-crimped wires. This also eliminates the discussion of
when to conformal coat: Before or after wire soldering.

One glueing work cycle has been eliminated as we nolonger use the
suspended torus-style inductors.

Inspection (and the occasional replacement) of power deviced in the
boost converters is greatly improved by not hiding components under the
huge inductors covered in epoxy. All components are now freely visible
and can be inspected and indeed replaced in case of a fault.

The kill-switch circuit can now be configured to ``ISIS standard
lock-less'' without hacking the circuit.

\subsection{Aesthetics}

A high degree of uniformity between similarly functioning sub-part of
the board has been implemented resulting in an appearance which is
``easy on the eye'' and gives a quick overview of block functionality.

Tracks and planes on the outside out the board are minimised to yield a
less confusing appearance.

The screws holding the batteries in place will be counter sunk in the
bottom side of the board giving a more finished look. It also makes the
screw head glueing look less messy.

Any sharp corners are rounded to blend into the style of the rest of the
Nano product family.

\subsection{Change Overview}

\begin{itemize}
\item
  Current measurement on all three PV inputs
\item
  Current measurement on all six lath-up protected outputs
\item
  Optimised latch-up detection
\item
  Power cycling ability added to kill-switch. Controlled by uC
\item
  Connection matrix enabling any of the two buck converters to feed any
  of the six latch-up protected output.
\item
  Added ISP PicoBlade connector to board edge
\item
  Added standard GOSH-enabled serial port to board edge
\item
  Batteries rotated 180 degrees to give better access to arm jumper
\item
  Battery tabs connected to board using slotted holes for added
  robustness
\item
  Boost converters use smaller SMD inductors for added effeciency and
  more robust assembly
\item
  Boost converter power devised are nolong hidden under epoxy
\item
  Buck converters use smaller inductors for higher effeciency
\item
  PV inputs now use 4-pin PicoBlade connectors for easy integration and
  wire redundancy mating 1-1 with new P110 solar panels
\item
  Kill-switch connections are now through two 2-pin PicoBlades placed at
  the cornes where ISIS kill-switches are located
\item
  Flight preperation panel connection is now a 6-pin PicoBlade
\item
  Microprocessor is upgraded to version with more ADC channels to
  fasilitate the added current measurements. The core is unchanged
  (Atmega 8)
\item
  The software will nolonger be a custom monolitic blop which has been
  hard to maintain. The new processor has more memory and enables the
  use of standard GomSpace libraries including GOSH. This will help keep
  the P31U up to date as various software libraries mature.
\item
  Battery screw heads are countersunk
\item
  Board corners are rounded
\item
  Old un-used IO connections to the CSK connectors are removed (only
  used on the the very first P30 built in 2007).
\item
  Power point tracking will be able to control the input voltage in the
  entire range from 0V to a given max instead of the current system
  which can only increase the voltage relative to a fixed fall-back
  level (usually called the hardware power point)
\item
  Overall quality appearance is improved through strict design and
  layout philosophy
\end{itemize}

\begin{table}[b]
\centering
\begin{tabular}{| >{\centering\arraybackslash}m{13cm} |}
\hline
GomSpace ApS  Niels Jernes Vej 10  9220 Aalborg E  Denmark\newline
Tel: +45 9635 6111  Fax: +45 9635 4599  Web: www.gomspace.com
\newline \newline
The information furnished by GomSpace in this data sheet is believed to be accurate and reliable. However, no responsibility is assumed by GomSpace for its use. GomSpace reserves the right to change circuitry and specifications at any time without notification to the customer. Current sale and delivery conditions can be downloaded from the GomSpace homepage.
\newline \newline
2012 GomSpace ApS.\\
\hline
\end{tabular}
\end{table}


\end{document}
